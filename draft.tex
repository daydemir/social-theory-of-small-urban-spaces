\documentclass[12pt]{article}

% Packages
\usepackage[utf8]{inputenc}
\usepackage[T1]{fontenc}
\usepackage{crimson}
\usepackage{geometry}
\usepackage{setspace}
\usepackage{graphicx}
\usepackage{booktabs}
\usepackage{natbib}
\usepackage{hyperref}
\usepackage{amsmath}
\usepackage{parskip}

% Page setup
\geometry{margin=1in}
\doublespacing
\setlength{\parskip}{1.5em}
\hyphenpenalty=10000
\exhyphenpenalty=10000
\emergencystretch=3em

% Title information
% \title{Ambient Co-Presence and the Social Imagination of Small Urban Spaces}
\title{Ambient Co-Presence and Social Imagination in Urban Spaces}
\title{\huge Ambient Co-Presence \\ \large and\\ \vspace{-1em} \huge Social Imagination in Small Urban Spaces}
\author{Deniz Aydemir}
\date{\today}

\begin{document}

\maketitle

\subsection*{Towards a generalization of people watching}
Most of our social experiences are not social interactions. We frequently see, walk by, sit near, and notice others. But we only interact with a small subset of those we are near. We might call this form of non-interactive proximity \textit{ambient co-presence}.

In cities, the divergence between those with whom we share ambient co-presence and those with whom we actually interact is especially pronounced. We might rub shoulders with hundreds of people for every one interaction. On a walk down a busy street in New York City, that could become thousands.

Much of existing social theory attempts to describe how our interactions and relationships shape the way we think about community, norms, and belonging. In recent years, there have been more explorations of how ambient or fleeting modes of social experience develop our social conceptions (see: Zahnow, Blokland). But these explorations underspecify the diversity of ambient social experiences, and don't explain the mechanism for their effects. 

Our goal here will be to (1) provide a typology of \textit{ambient co-presence} that can be used to add greater precision to discussions of non-interactive social experiences, and (2) provide a hypothesis for how our imagination uses ambient co-presence in shared spaces to impact our sense of community and belonging. 

These non-interactie, fleeting social experiences are foundational in city life, and understanding them better may help us create healthier communities in our cities.

\section{A Typology of Co-Presence}

We can describe a moment where co-presence occurs between two individuals as an \textit{event} where the two individuals are in perceptible distance to each other. If either of the two individuals can see, hear, smell, or touch each other we will consider them to be in co-presence. 

We can then divide these events into two types we've already described: co-presence that involves an interaction (maybe we can call this \textit{engaged} co-presence) and co-presence that does not, which we call \textit{ambient}. This division is not always clear. How many words shared constitutes an interaction? Is eye contact an interaction? Is wordlessly holding a door for someone else an interaction?

For our purposes, we should focus only on co-presence that can be uncontroversially defined as non-interactive. This means that the strongest co-presence we will consider ambient is an event where person A consciously observes person B, and person A has no knowledge of whether B reciprocated in any conscious noticing of A. Considering ambient co-presence this way allows us to guarantee that any impact on A from the event is irrespective of B's actions. If A consciously sees B acknowledging or responding to A in any way, then we will not consider that event to be one of ambient co-presence. For example, if A and B make eye contact their interaction is no longer only ambient.

This is likely too strong. When you make room for someone to sit next to you on a bus, you may interact with body language or eye contact. But that still seems to qualify as an ambient event, especially if the majority of your time in proximity is spent sitting next to each other without acknowledgment or any further communication. But we won't deal with such edge cases today. 

The operative concept we will use to qualify ambient co-presence is that the whole event is experienced inside A's own mind, and any impacts on A occur solely in A's imagination. In a purely ambient co-presence event, nothing is enacted or tested outside the mind. B does not offer anything other than their ambient effects on A's senses, no non-ambient form of meaningful communication. Despite this, these ambient encounters do shape the way we think. This power of ambient co-presence is what we are trying to tease out.

Because this ambient co-presence is a subjective experience, we can also differentiate between the observer individual (A) and the object individual (B) of the event. When we speak of the impacts of the experience, we will assume those impacts are only relevant to the observer. Naturally, B can be having a simultaneous ambient co-presence event where B is the observer and A is the object, but because it is not a tangible interaction the two simultaneous events may have no relevance to each other.

\subsection{Conscious and Unconscious}

We have described a one-sided, conscious observation as an example of an ambient co-presence event. But we can also include the unconscious or subconscious noticing of others as well. Ambient co-presence could also be described by B sitting in A's peripheral vision, with no need for A to consciously observe or even acknowledge to themselves the presence of B. This allows us to differentiate two types of ambient co-presence: \textit{conscious} and \textit{unconscious}.

\subsection{Familiar and Unfamiliar}

But this is not the only way we can differentiate types of ambient co-presence. In an ambient co-presence event, the observer may recognize the subject, perhaps on a commuter train where riders might often ride with familiar faces, or at a bar where others also frequent. So we can differentiate between \textit{familiar} and \textit{unfamiliar} ambient co-presence.

Familiarity here needs some explanation. Are repeated, conscious, ambient co-presence events between two individuals still ambient if they recognize each other?

As long as the two participants have not interacted before, we will consider repeated ambient co-presence events to be very much possible. It is natural that familiar strangers would be more likely to interact than unfamiliar strangers, and there are certainly interesting things to be said about the kinds of interactions that familiar strangers might have and the implications of those social interactions. For now, we will only be concerned with the impacts of familiar strangers who do not cross the threshold to engaging each other.

\subsection{Groups}

We may often be in public with others we know: our family, friends, colleagues. There are certainly different capacities for perceiving ambient co-presence when with a group compared with when one is alone. But there 

Discussing 

\subsection{Summary of Types}

% TODO: Create table of types of ambient co-presence
% Table should cross conscious/unconscious with familiar/unfamiliar

\begin{table}[h]
\centering
\begin{tabular}{lcc}
\toprule
 & \textbf{Familiar} & \textbf{Unfamiliar} \\
\midrule
\textbf{Conscious} & & \\
\textbf{Unconscious} & & \\
\bottomrule
\end{tabular}
\caption{Types of Ambient Co-Presence}
\label{tab:types}
\end{table}

There is no doubt that ambient co-presence is a pervasive part of the human experience. 

Sure, we can imagine small towns with tight-knit communities where almost no co-presence goes unengaged -- everyone acknowledges one another at every opportunity. Perhaps this is the kind of community humans evolved to live in. But this is not what social life looks like in most places today, and certainly not in cities. 

\section{A Small World of Urban Spaces}

% Interactions can be used to

% ambient co-presence creates a mental model for the world


In cities, we share many spaces with others we never interact with. Let's describe three examples that we can use to explore the different types of ambient co-presence and their implications.

\subsection{Bea sits in a bar}
Bea decides to grab a drink at a local bar she's been meaning to visit. She sits at the bar, interacts with the bartender, but otherwise keeps to herself. There are some individuals, some couples, and some groups of three or four scattered around the bar. In addition to the bartender, there are two servers who are attending those sitting at tables. 


% This is the quintessential third place as (name?) initially defined them. We can see the similarities to conceptions of public space put forward by Habermas  

% But why bar?
% - third places
% - why bar and not cafe? in order to highlight the maximum amount of eavesdropping possible

\subsection{Sue walks down a street}
Sue is walking down a crowded street at the end of the workday. She sees many people walking at various degrees of urgency, many people standing, some shopkeepers, some loitering. She sees people turning onto smaller streets, she sees people who look lost, and she sees people going in and out of stores.

\subsection{Trina commutes on a train}
Trina always takes the commuter rail to work, this morning is no different. She sees the usual crowd she usually sees, along with some she doesn't recall seeing before. She sits at an empty pair of seats, but a couple stops later an unfamiliar stranger sits next to her.

We've laid out the types of ambient co-presence and a small world of social experiences. Now let's explore some hypotheses.

\section{Hypotheses for Ambient Co-Presence in Small Urban Spaces}

- familiarity without strong interaction can create belonging and cohesion
- we dont need weak ties or closure (coleman / granovetter)

so why can these types of 



\subsection{Urban community beyond network ties}

Granovetter's concept of network ties is useful to help describe many types of social phenomena. But when it comes to forming community and belonging, network ties do not explain the whole story.

We do not need strong local networks to feel belonging in our neighborhood. The role of familiar strangers 


familiar ambient co-presence



\subsection{Urban community through imagination}

Ambient co-presence doesn't create network ties, but it activates our imagination.

We hypothesize here that the mechanism that creates a sense of comfort, belonging, and social cohesion is the imagined connection and camaraderie we create through ambient co-presence events. Familiar ambient co-presence is especially well-justified to qualify for this hypothesis. We know that recognizing strangers, even those with whom we have no interaction, leads to feelings of security, safety, and comfort (cite: Zahnow). 

This makes sense. Familiarity breeds comfort. But what is the mechanism that converts ambient familiarity into comfort? And is ambient familiarity really sufficient or necessary to foster belonging?

\subsubsection{With familiar ambient co-presence}

Zahnow and Corcoran see feelings of belonging and safety grow both when observers experience a reciprocation of familiarity (a nod from a stranger) and when they don't. Zahnow and Corcoran believe that recognition creates a reciprocal signal of acceptance in the in-group, but they don't explain that more than to say that the observer experiences a "sense of shared, symbolic identification" (cite: Zahnow security). 


So Zahnow and Corcoran posit that the observer's imagination is doing the heavy-lifting of creating a "symbolic identification", but they underemphasize that this feeling of acceptance does not require any acknowledgment from the familiar object individual. Ambient familiarity alone triggers our imagined sense of community.

We can think here of Trina. She sees familiar faces on her train, and she might feel a sense of camaraderie for their shared experience. She may notice when someone who's usually on the commute isn't one day, even if they've never interacted. She may wonder if they're on vacation, or if something worse happened. 

This imagined community we create in our minds through ambient familiarity can be related to Anderson's concept of the nation as an imagined community. In Anderson's structure, a nation can create an imagined community using shared media and narratives with millions of people we will never interact (cite: Anderson). At the urban scale, we see that familiarity and shared spaces creates an imagined community of people even if we never interact with them. 

But do we need familiarity to feel belonging? Or can we feel familiarity not only through repeated co-presence and by recognizing faces, but in other ways?

\subsubsection{With unfamiliar ambient co-presence}

Ambient familiarity 


City-dwellers identify with their city, and imagine a community of their city. 

Let's say Sue is a New Yorker, and she's walking down 5th Avenue. Sue may not recognize a single person, but still feel a sense of belonging and community simply because the people with whom she experiences ambient co-presence share this space she calls home. 

Perhaps if everyone looks like a tourist, Sue might feel like she is not in the presence of their community. She may not feel belonging in that moment. But, of course, she will not always know for sure if every person she walks by is in fact a New Yorker. She will use her imagined concept of what her city's community is (or should be), and negotiate that with what she observes to determine if she is in the presence of her community. 

This cognitive urban community, the negotiation of that mental image with those real object individuals of the ambient co-presence, and the subsequent feeling of belonging all occur within the observer's imagination. It does not require interaction, and it does not require familiarity.

This community imagination is not only active when considering a community of a city. Consider 

- garnette



% garnette?
% gemeinschaft vs gesellschaft ?

% imagined communities / solidarity
% - familiar events create solidarity
% - being in an ethnically familiar place

\subsection{Ambient norm enforcement}

% what about impact on norms, coleman
% - norms are enforced not through any two-sided interaction, but through a series of ambient interactions (no two people necessarily need to make eye contact for a norm to be enforced)
% - if someone is playing loud music on a train
%    - i may see others' reactions, and see that as enforcement
%    - it may bother me, and thus through ambient co-presence i derive my own categorical imperative

The chain of ambient co-presence events

\section{Role of the shared urban space}

- other potential benefits of ambient co-presence, especially unfamiliar 
- provide places where people can recognize and be recognized
- mental health benefits, polarization


% OTHER OPTIONS:
% \subsection{Aesthetics and Fashion}

% % no question about impact on aesthetics, see fashion, bourdieu

% \subsection{Mental Health Benefits}


% \section{Qualities of Ambient Co-Presence}

% % qualities of ambient co-presence
% % - ethnic proximity, etc

% \section{Ambient as a Precursor to Engaged Co-Presence}

% - children mimic parents

% ambient as a lead to engaged co-presence

% TODO: conclusion

\bibliographystyle{apalike}
\bibliography{references}

\end{document}
