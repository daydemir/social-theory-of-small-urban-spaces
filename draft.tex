\documentclass[12pt]{article}

% Packages
\usepackage[utf8]{inputenc}
\usepackage[T1]{fontenc}
\usepackage{crimson}
\usepackage{geometry}
\usepackage{setspace}
\usepackage{graphicx}
\usepackage{booktabs}
\usepackage{natbib}
\usepackage{hyperref}
\usepackage{amsmath}

% Page setup
\geometry{margin=1in}
\doublespacing
\sloppy

% Title information
\title{\huge Ambient Co-Presence \\ \large and a \\ \vspace{-1em} \huge Social Theory of Small Urban Spaces}
\author{Deniz Aydemir}
\date{\today}

\begin{document}

\maketitle

% \section*{Turning to the Unspoken}
Most of our social experiences are not social interactions. We frequently see, walk by, sit near, and notice others. But we only interact with a small subset of those we are near. We might call this form of non-interactive proximity \textit{ambient co-presence}.

In cities, the divergence between those with whom we share ambient co-presence and those with whom we actually interact is especially pronounced. We might rub shoulders with hundreds of people for every one interaction. On a walk down a busy street in New York City, that could become thousands.

Much of existing social theory describes how our interactions and relationships shape the way we think about community, norms, and belonging. In recent years, there has been more exploration of how non-interactive modes of social experience develop our social conceptions (see: Zahnow, Blokland). But these explorations underspecify the diversity of ambient social experiences, and underestimate the strength of their effects. 

Our goal here will be to (1) provide a typology of \textit{ambient co-presence} that can be used to add greater precision to discussions of non-interactive social experiences, and (2) hypothesize ways that ambient co-presence in shared spaces might have under-explored impacts on our social thinking.

\section{A Typology of Co-Presence}

We can describe a moment where co-presence occurs between two individuals as an \textit{event} where the two individuals are in perceptible distance to each other. If either of the two individuals can see, hear, smell, or touch each other we will consider them to be in co-presence. 

We can then divide these events into two types we've already described: co-presence that involves an interaction (maybe we can call this \textit{engaged} co-presence) and co-presence that does not, which we call \textit{ambient}. This division is not always clear. How many words shared constitutes an interaction? Is eye contact an interaction? Is wordlessly holding a door for someone else an interaction?

For our purposes, we should focus only on co-presence that can be uncontroversially defined as non-interactive. This means that the strongest co-presence we will consider ambient is an event where person A consciously observes person B, and person A has no knowledge of whether B reciprocated in any conscious noticing of A. Considering ambient co-presence this way allows us to guarantee that any impact on A from the event is irrespective of B's actions. If A consciously sees B acknowledging or responding to A in any way, then we will not consider that event to be one of ambient co-presence. For example, if A and B make eye contact their interaction is no longer only ambient.

This means the whole event is experienced inside A's own mind, and any impacts on A occur solely in A's imagination. In a purely ambient co-presence event, nothing is enacted or tested outside the mind. Still, these encounters do shape our perception. We can also then differentiate between the \textit{observer} (A) and the \textit{subject} (B) in any ambient co-presence event.

\subsection{Conscious and Unconscious}

We have described a one-sided, conscious observation as an example of an ambient co-presence event. But we can also include the unconscious or subconscious noticing of others as well. Ambient co-presence could also be described by B sitting in A's peripheral vision, with no need for A to consciously observe or even acknowledge to themselves the presence of B. This allows us to differentiate two types of ambient co-presence: \textit{conscious} and \textit{unconscious}.

\subsection{Familiar and Unfamiliar}

But this is not the only way we can differentiate types of ambient co-presence. In an ambient co-presence event, the observer may recognize the subject. Perhaps on a commuter train where riders might often see familiar faces, or at a bar where others also frequent. So we can differentiate between \textit{familiar} and \textit{unfamiliar} ambient co-presence.

\subsection{Summary of Types}

% TODO: Create table of types of ambient co-presence
% Table should cross conscious/unconscious with familiar/unfamiliar

\begin{table}[h]
\centering
\begin{tabular}{lcc}
\toprule
 & \textbf{Familiar} & \textbf{Unfamiliar} \\
\midrule
\textbf{Conscious} & & \\
\textbf{Unconscious} & & \\
\bottomrule
\end{tabular}
\caption{Types of Ambient Co-Presence}
\label{tab:types}
\end{table}

% This is not a
\section{Small Urban Spaces}

% Interactions can be used to

% ambient co-presence creates a mental model for the world

There is no doubt that ambient co-presence is a pervasive part of the social experience of all humans. We can imagine small towns with tight-knit communities where almost no co-presence goes unengaged -- everyone acknowledges one another at every opportunity. But this is not life in cities. 

% gemeinschaft vs gesellschaft 

\section{Impacts of Ambient Co-Presence}

\subsection{Aesthetics and Fashion}

% no question about impact on aesthetics, see fashion

\subsection{Norms and Social Enforcement}

% what about impact on norms, coleman
% - norms are enforced not through any two-sided interaction, but through a series of ambient interactions (no two people necessarily need to make eye contact for a norm to be enforced)
% - if someone is playing loud music on a train
%    - i may see others' reactions, and see that as enforcement
%    - it may bother me, and thus through ambient co-presence i derive my own categorical imperative

\subsection{Imagined Communities and Solidarity}

% imagined communities / solidarity
% - familiar events create solidarity
% - we know benefits to polarization

\subsection{Mental Health Benefits}

% mental health benefits

\section{Qualities of Ambient Co-Presence}

% qualities of ambient co-presence
% - ethnic proximity, etc

\section{Ambient as a Precursor to Engaged Co-Presence}

% ambient as a lead to engaged co-presence

\section{Conclusion}

% TODO: conclusion

\bibliographystyle{apalike}
\bibliography{references}

\end{document}
